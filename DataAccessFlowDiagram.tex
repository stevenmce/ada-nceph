% Created 2012-10-19 Fri 11:43
\documentclass[a4paper]{article}
\usepackage[utf8]{inputenc}
\usepackage{hyperref}
\usepackage{graphicx}
\usepackage{longtable}
\usepackage{float}
\usepackage{pdfpages}
\providecommand{\alert}[1]{\textbf{#1}}

\title{ADA-NCEPH Data Access Procedure}
\author{Ivan Hanigan and Steven McEachern}
\date{\today}
\hypersetup{
  pdfkeywords={},
  pdfsubject={},
  pdfcreator={Emacs Org-mode version 7.8.11}}

\begin{document}

\maketitle

% Org-mode is exporting headings to 3 levels.
\tableofcontents

\clearpage
\section{TODOLIST}
\label{sec-1}
\subsection{\textbf{TODO} Ivan send graphivis lowlevel versions to Steve by noon Fri (also NCEPH policy)}
\label{sec-1-1}
\subsection{\textbf{TODO} Steve review and comment}
\label{sec-1-2}
\subsection{\textbf{TODO} Ivan to revise lucidchart highlevel version on Mon-Tues}
\label{sec-1-3}
\subsection{\textbf{TODO} Ivan finalise and send to BDM by Wed-ish, CC Steve}
\label{sec-1-4}
\section{Introduction}
\label{sec-2}

The aim of this document is to describe the procedure for accessing restricted health data through the proposed ANU Secure Data Hub, administered by the ADA and NCEPH.

The following descibes procedures and processes for three different agents in the system, with different roles:
\begin{itemize}
\item Users,
\item User Administrators, and
\item Data Administrators.
\end{itemize}

The User and Data information that is used to control the actions of the system are stored in a Database at ANU referred to as the ANU-User-DB.
\newpage
\section{Getting Access}
\label{sec-3}

The procedure to help users apply for and gain access to these data is a set of formally defined steps that are designed to move the User from an initial state of searching for data through the process of gaining Ethics Approval from a Human Research Ethics Committee, and Project Level Approval from the Registrar of Births, Deaths and Marriages.  Then a confidentialised (often aggregated) dataset is provided in a secure manner as appropriate given the nature of the data and any project management related criteria such as access to remote secure servers versus encrypted archives accessed on local disk media.
\subsection{Flow Chart of Steps to Get Access}
\label{sec-3-1}
\subsubsection{\textbf{TODO} change this to the lucidchart version}
\label{sec-3-1-1}

\begin{figure}[!h]
\centering
\includegraphics[width=\textwidth]{DataAccessFlowDiagram-GettingAccess.pdf}
\caption{Flow Diagram of Getting Access}
\label{fig:DataAccessFlowDiagram-GettingAccess}
\end{figure}
\clearpage
\section{Managing Access}
\label{sec-4}

The User Administrator is responsible for managing current access.  This means they regularly run a query on the ANU-User-DB to make a list of all Projects and Users and each Project is sent a reminder to report any changes in Project Status.  These reminders are sent annually to coincide with a similar reminder sent by the Ethics Committee.  The purpose of these reminders is to ensure that Project management plans continue to consider data security as a primary concern, even during long multi-year projects where many project management and staffing issues inevitably arise.
\subsection{Flow Chart of Steps to Manage Access}
\label{sec-4-1}
\subsubsection{\textbf{TODO} change this to the lucidchart version}
\label{sec-4-1-1}

\begin{figure}[!h]
\centering
\includegraphics[width=\textwidth]{DataAccessFlowDiagram-ManagingAccess.pdf}
\caption{Flow Diagram of Managing Access}
\label{fig:DataAccessFlowDiagram-ManagingAccess}
\end{figure}
\clearpage
\section{Ending Access}
\label{sec-5}

The procedure for ending access aims to ensure that data are both securely and sustainably stored.  It is very important that files used for authorised projects are never re-used in un-authorised projects, but that future researchers may have the opportunity to create an authorised project and potentially replicate historical analyses.  This is an important part of reproducible research and the robust practice of scientific enquiry.
\subsection{Flow Chart for Ending Access}
\label{sec-5-1}
\subsubsection{\textbf{TODO} change this to the lucidchart version}
\label{sec-5-1-1}


\begin{figure}[!h]
\centering
\includegraphics[width=\textwidth]{DataAccessFlowDiagram-EndAccess.pdf}
\caption{Flow Diagram for Ending Access}
\label{fig:DataAccessFlowDiagram-EndAccess}
\end{figure}
\clearpage

\end{document}